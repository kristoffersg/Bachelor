\thispagestyle{fancy}
\chapter{Systembeskrivelse}
\label{chp:systembeskrivelse}
På baggrund af projektafgrænsningen vil dette kapitel præsenterere det system, som udgør rammerne for systemets udvikling. I figur \ref{fig:systembeskrivelse} forsøges det afgrænsede projekt beskrevet. Systemet beskrives som en mobilapplikation kørende på flere Windows Phone 8 enheder, en serverapplikation kørende på en Windows 7 eller nyere og en database, hvor relevante data opbevares. Det endelige system udvikles som et proof of concept til festivaller og navngives \textit{Mingle}.

\figur{systembeskrivelse.jpg}{Beskrivelse af systemet}{fig:systembeskrivelse}{1}

Det er målet igennem en mobilapplikation at støtte festivalgæsterne i socialisering og skabelse af nye venskaber. Dette gøres med et spørgsmålsspil, hvor to personer i samarbejde skal svare på spørgsmål om hinanden. Mobilapplikationen, serverapplikationen og databasen giver tilsammen denne personlige spiloplevelse.

For at skabe en personliggørelse af spillet er systemet tæt koblet med Facebook, hvorfra relevante informationer om brugerne hentes og gemmes i databasen. Når en bruger ønsker at logge ind benyttes email og adgangskode til Facebook. Hvis en bruger ønsker at benytte Mingle, er en Facebook bruger derfor et krav. 

Når et spørgsmålsspil startes fremvises det samme spørgsmål på mobilapplikationerne. Disse spørgsmål modtages fra serveren, som henter dem fra databasen. På databasen opbevares både spørgsmål og tilhørende svarmuligheder.

Ved et gennemført spil optjenes point. Disse point gemmes ligeledes i databasen og kan benyttes til at købe varer på festivaller. En festivalmedarbejder  verificerer købet, og pointene trækkes fra brugerens totale antal point.


Da Mingle har fokus på at skabe nye venskaber, sættes en begrænsning, hvor der udelukkende tildeles point for første spil som i samarbejde er vundet. Dette realiseres igennem en historik, som serveren håndterer og gemmer på databasen.






 

 