
%%%%%%%%%%%%%%%%%%%%%%%%%%%%%%%%%%%%%%%%%%%%%%%%%%%%%%%%%%%%%%%%%%%%%%%%%%%%%
%                              Custom functions                             %
%                         For general use in projects                      %
%%%%%%%%%%%%%%%%%%%%%%%%%%%%%%%%%%%%%%%%%%%%%%%%%%%%%%%%%%%%%%%%%%%%%%%%%%%%%

% For use in definition of new commands:
\usepackage{xspace}           % \newcommand{\cmd}{text\xspace} ensures the correct spacing between text and other words/characters.

% For cellheight adjustments in nomenklatur-function
\usepackage{cellspace}
    \addtolength\cellspacetoplimit{0pt}
    \addtolength\cellspacebottomlimit{0pt}


%
%% Custom "paragraph" function for making line-seperated paragraphs (e.g. double newline)
%

% Make double line break with no line indent (for changing the way paragraphs are made)
\newcommand{\newpar}{$\,$ \\ $\,$ \\}
% Alias of \newpar
\newcommand{\newparagraph}{\newpar{}}

\newcommand{\subsubsubsection}[1]{%
\newpar\paragraph{#1}$\,$\\%
}


%
%% Function for creating horizontal lines between points in a recipe-like list
%
\newcommand{\recipeline}{%
\vspace{2pt}
\hrule
\vspace{2pt}
}
%
%% Function for creating a boxed, centered equation  behaving like the equation environment
%
\newcommand{\boxedeq}[2]{%
	\[\fbox{%
			\addtolength{\linewidth}{-2\fboxsep}%
			\addtolength{\linewidth}{-2\fboxrule}%
			\begin{minipage}{#2}%
				\bigskip%
				\[#1\]%
				\hspace{1cm}%
			\end{minipage}%
	}\]%
			\addtolength{\linewidth}{-2\fboxsep}%
			}
			%
			%% Function for creating a boxed, centered equation  behaving like the equation environment
			%
			\newcommand{\boxedeqlabel}[3]{%
				\[\fbox{%
			\addtolength{\linewidth}{-2\fboxrule}%
			\begin{minipage}{#2}%
				\bigskip%
				\begin{equation}#1\end{equation}%
				\label{#3}%
				\hspace{1cm}%
			\end{minipage}%
	}\]%
}

%
%% Environment for displaying a 0th, 1st or 2nd order tensor in matrix format
%
\newenvironment{tensor}
{
	\begin{pmatrix}
}
{
    \end{pmatrix}
}

%
%% For displaying vectors
%
\newenvironment{vect}
{
	\begin{Bmatrix}
}
{
    \end{Bmatrix}
}

% VECTOR IS DEFINED
\newcommand{\vectq}[1]{\ensuremath{%
%	\{#1\}
\text{\overrightharp{$#1$}}
}}

% UNIT VECTOR IS DEFINED
\newcommand{\unitvecq}[1]{\ensuremath{%
\widehat{#1}
}}

% TIME INVARIANT IS DEFINED
\newcommand{\tInvar}[1]{
\widetilde{#1}
}
%
%% For displaying matrices
%
\newenvironment{matr}
{
	\begin{bmatrix}
}
{
    \end{bmatrix}
}
\newcommand{\matrq}[1]{\ensuremath{%
%	[#1]
\underline{#1}
}}
\renewcommand{\Re}[1]{
%\text{Re}\Big\langle#1\Big\rangle%
\text{Re}\left\{#1\right\}
}
\renewcommand{\Im}[1]{
%\text{Im}\Big\langle#1\Big\rangle%
\text{Im}\left\{#1\right\}
}
%
%% Function for creating small introductions for every new chapter (italic text)
%
\newcommand{\chapintro}[1]{%
    \textit{#1} \\ \linebreak%
    \noindent%
}

%
%% Function for creating small introductions for every new section (italic text)
%
\newcommand{\secintro}[1]{%
    \textit{#1} \\ \linebreak%
    \noindent%
}

%
%% Function for creating isolated remarks (indented italic text)
%
\newcommand{\remark}[1]{%
	\begin{center}%
		\parbox{0.8\textwidth}{\textit{#1}}%
	\end{center}%
    \noindent%
}

%
%% Function for creating isolated remarks (indented italic text)
%
\newcommand{\makequote}[2]{%
	\begin{center}%
		\parbox{0.8\textwidth}{"\textit{#1}" #2}%
	\end{center}
    \noindent%
}

%%% Function for using \verytightlist - not very pretty!
%\newcommand{\verytightlist}{%
%    \setlength{\itemsep}{0.05cm} \setlength{\parskip}{1pt}}
%
%
%%% Function for almost using \tightlist without using the memoir-class
%\newcommand{\allmosttightlist}{%
%    \setlength{\itemsep}{0.1cm} \setlength{\parskip}{1pt}}
%
%
%%% Function for using \semitightlist
%\newcommand{\semitightlist}{%
%    \setlength{\itemsep}{0.2cm} \setlength{\parskip}{1pt}}
%

%
%% Comments (shown in document outer margin)
%
\renewcommand{\comment}[2]{%
    \textcolor{blue}{\textbf{*}} \marginpar{%
        \fcolorbox{white}{blue}{\parbox{3.5cm}{%
            \color{white} \small \sffamily \textbf{#1:} #2}%
        }%
    }%
}
%\newcommand{\comment}[2]{} % Use for hiding comments

%
%% Fixmes (shown in document outer margin)
%
\renewcommand{\fixme}[2]{%
    \textcolor{red}{\textbf{!}} \marginpar{%
        \fcolorbox{white}{red}{\parbox{3.5cm}{%
            \color{white} \small \sffamily \textbf{FIXME:} #2 (#1)}%
        }%
    }%
}
%\newcommand{\fixme}[2]{} % Use for hiding fixmes


%
%% Pretext style headers/footers (for use before \mainmatter}
%% NB: Use only for report/book style in documentclass
%
\newenvironment{pretext}
{
	% Don't use
	\pagestyle{plain}
}
{
	\cleardoublepage
	\pagestyle{fancy}
}


%%% Function for using \bf to write in bold
\renewcommand{\bf}{%
    \bfseries%
}

%%% Function for using \it to write in italic
\renewcommand{\it}{%
    \itshape%
}


%% Home made commands by Rene
\newcommand{\abs}[1]{\lvert#1\rvert}
\newcommand{\sgn}[1]{\text{sgn}\left(#1\right)} %Sign function
\newcommand{\nrcs}[1]{\: \put(5.5,3.5){\circle{14}}#1 \,} %Laver en cirkel omkring et to cifferet tal, nummer cirkel stor fx \nrcs{15}
\newcommand{\nrc}[1]{\: \put(3,3.5){\circle{12}}#1 \,} %Laver en cirkel omkring et �t cifferet tal, nummer cirkel fx \nrc{5}
\newcolumntype{R}[1]{>{\raggedleft\arraybackslash}p{#1}} %R{2cm} laver en right orienteret s�jle med en bredde p� 2cm






%%%%%%%%%%%%%%%%%%%%%%%%%%%%%%%%%%%%%%%%%%%%%%%%%%%%%%%%%%%%%%%%%%%%%%%%%%%%%
%                              Custom functions                             %
%                       Primarily for this project; I6                      %
%%%%%%%%%%%%%%%%%%%%%%%%%%%%%%%%%%%%%%%%%%%%%%%%%%%%%%%%%%%%%%%%%%%%%%%%%%%%%

%
%% Adds vertical space around \hline in tabular, T = top
%
\newcommand\T{%
    \rule{5pt}{5ex}%
}

%
%% Adds vertical space around \hline in tabular, B = buttom
%
\newcommand\B{%
    \rule{0pt}{2.6ex}%
}

%
%% Nomenclature below equations
%
\newcommand\nomenklaturCellHeight{%
\includegraphics[scale=1]{styles/nomenklaturCellHeight.PNG}%
}
\renewcommand\nomenklaturCellHeight{}


\newcommand\nomenklatur[2]{%
\nomenclature{$#1$}{#2.}{}{}%
\begin{minipage}[top]{0.7\textwidth}\flushleft%\renewcommand\arraystretch{1.25}% (Value=1.0 is for standard spacing}%
\begin{tabular}{S{p{1.0cm}}S{p{1.3cm}} Sl}%
{$\,$} & \ensuremath{#1} &\nomenklaturCellHeight\parbox{12cm}{\footnotesize{#2.}}%
\end{tabular}%
\end{minipage}%
%}
\\ \noindent%
}

%
%% Start nomenclature below equations - use this command for the first line of each nomenclature-block.
%
\newcommand\nomenklaturstart[2]{%
\nomenclature{$#1$}{#2.}{}{}%
\begin{minipage}[top]{0.7\textwidth}\flushleft%\renewcommand\arraystretch{1.25}% (Value=1.0 is for standard spacing}%
\begin{tabular}{>{\setlength{\parindent}{-0.2cm}}S{p{1.0cm}}S{p{1.3cm}} Sl}
where:\nomenklaturCellHeight & \ensuremath{#1} &\nomenklaturCellHeight\parbox{12cm}{\footnotesize{#2.}}%
\end{tabular}%
\end{minipage}%
%}
\\ \noindent%
}

%%%%%%%%%%%%%%%%%%%%%%%%%%%%%%%%%%%%%%%%%%%%%%%%%%%%%%%%%%%
%%figur nomenklatur
%%%%%%%%%%%%
%\newcommand\ehnomenklatur[3]{%
%\noindent
%\nomenclature{$#1$}{#2}{}{}%
%\begin{minipage}[top]{0.7\textwidth}\flushleft%\renewcommand\arraystretch{1.25}% (Value=1.0 is for standard spacing}%
%\begin{tabular}{S{p{1.0cm}} S{p{1.3cm}} Sl}%
%{$\,$} & \ensuremath{#1} &\nomenklaturCellHeight\parbox{8cm}{\footnotesize{#2}} \ensuremath{\footnotesize{#3}}%
%\end{tabular}%
%\end{minipage}%
%%}
%\\ \noindent%
%}
%
%%
%%% Start nomenclature below equations - use this command for the first line of each nomenclature-block.
%%
%\newcommand\ehnomenklaturstart[3]{%
%\noindent
%\nomenclature{$#1$}{#2}{}{}%
%\begin{minipage}[top]{0.7\textwidth}\flushleft%\renewcommand\arraystretch{1.25}% (Value=1.0 is for standard spacing}%
%\begin{tabular}{>{\setlength{\parindent}{-0.2cm}}S{p{1.0cm}}S{p{1.3cm}} Sl}
%$\quad$ \\
%where:\nomenklaturCellHeight & \ensuremath{#1} &\nomenklaturCellHeight\parbox{8cm}{\footnotesize{#2}} \ensuremath{\footnotesize{#3}}%
%\end{tabular}%
%\end{minipage}%
%%}
%\\ \noindent%
%}


%
%% Command for definition of new terms in text
%
\newcommand\term[1]{%
    \emph{#1}%
}

%% Command for creating a double-bar over a matrix symbol
\newcommand\dbar[1]{%
    \ensuremath{\bar{\bar{#1}}}%
}

%% Command for definition of planar cross product
\newcommand\planarCross{%
%    \ensuremath{\underline{\times}}%
%    \ensuremath{\dtimes}%
\begin{turn}{90}\ensuremath{\ltimes}\end{turn}
}


