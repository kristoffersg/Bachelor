\thispagestyle{fancy}
\chapter{Opgavebeskrivelse}
Målet med projektet er i al sin enkelhed, at undersøge mulighederne for at lave EMG signal genkendelse på armens muskler. Hensigten er her at lave genkendelsesalgoritmer ud fr machnelearning principper. f.eks. som deep learning, til at skelne mellem håndbevægelser.\\

Det er i projektet, tanken at anvende et wearable device til opsamling af EMG data. Denne enhed er Myo \href{https://www.thalmic.com/en/myo/}{https://www.thalmic.com/en/myo/}, et EMG armbånd med 8 EMG sensore og 9 aksial IMU med accelaometer, gyroscope og magnetometer. EMG dataen skal bruges til at “træne” programmet/softwaren/alogritmerne til at skelne mellem forskellig håndbevægelse.\\

Projektet er inspireret af myoelektriske proteser som anvender EMG signaler til at styre elektriske armproteser. Det er således håbet at projektet vil kunne anvendes i realtime til at styre armproteser, robotarme, objekter i virtual reality eller lignende.

\begin{itemize}
	\item Anskaffelse af store mængder EMG data fra arm eller lign.
	\item Udvikle visuelt softwareinterface til dataprocessering.
	\item Udvikle machine learning algoritme til genkendelse af EMG data.
	\item Identificere mulige anvendelsesscenarier.
\end{itemize}